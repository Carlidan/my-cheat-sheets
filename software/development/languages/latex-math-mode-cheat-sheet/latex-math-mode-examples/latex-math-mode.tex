\documentclass[10pt]{article}
\usepackage[english]{babel}
\usepackage[utf8]{inputenc}
\usepackage{amsmath}
\usepackage{listings}

\title{\vspace{-5.0cm}LaTeX MATH MODE}
\author{\vspace{-0.1cm}Jeff DeCola}
\date{\vspace{-0.5cm}}

\setlength{\parindent}{0em} %Paragraph Indent
\setlength{\parskip}{1em} %Paragraph Spacing

\begin{document}
\maketitle

\begin{center}
\textbf{THREE WAYS TO DECLARE MATH MODE IN LATEX}
\end{center}

\textbf{INLINE MODE}

\begin{lstlisting}
This is einstein's equation inline $E=mc^2$.
\end{lstlisting}

This is einstein's equation inline $E=mc^2$.

\textbf{NEWLINE CENTER}

\begin{lstlisting}
$$
E=mc^2
$$
\end{lstlisting}

$$
E=mc^2
$$

\textbf{NEWLINE CENTER WITH NUMBER}

\begin{lstlisting}
\begin{equation}
	E=mc^2
\end{equation}
\end{lstlisting}

\begin{equation}
	E=mc^2
\end{equation}

\begin{center}
\textbf{FORMATTING MULTIPLE EQUATIONS IN MATH MODE}
\end{center}

\textbf{BEGIN GATHERED} (both equations centered)

\begin{lstlisting}
$$
\begin{gathered}
    a=b+c \\
    d+e=f
\end{gathered}    
$$
\end{lstlisting}

$$
\begin{gathered}
    a=b+c \\
    d+e=f
\end{gathered}    
$$

\textbf{BEGIN ALIGNED} (Both equation aligned on ampersand with tag)

\begin{lstlisting}
\begin{equation}
    \begin{aligned}
        a&=b+c \\
        d+e&=f
    \end{aligned}
\end{equation}
\end{lstlisting}

\begin{equation}
    \begin{aligned}
        a&=b+c \\
        d+e&=f
    \end{aligned}
\end{equation}

\end{document}
